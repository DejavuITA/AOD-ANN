\tikzsetexternalprefix{tikz/}	% set subfolder
\tikzsetnextfilename{NL_resonance}
\begin{tikzpicture}[baseline]
	\newcommand\CENTRAL{192}
	\newcommand\START{190.9}
	\newcommand\STOP {193}
	\newcommand\CC{299792458}
	
	\begin{axis}[
			axis x line*= bottom,
			xlabel = {Optical Frequency $\nu$ [$\si{\THz}$]},
			ylabel = {Transmission},
			legend columns=-1,
			legend cell align=right,
			legend style={ at={(0.5,-0.22)}, anchor=north },
			width=\textwidth*0.75,%
			height=207pt,
			xmin= 190.95, xmax = 193,
			%
			% global plot definition
			domain = \START:\STOP,
			samples = 101,
			smooth,
			no markers,
			cycle multi list={
%					exotic
					blue, orange, red
				},
			]
				
		\foreach \TA/\LA/\Radius/\Neff in {0.85/0.9/5/3.826, 0.85/0.9/5/3.827, 0.85/0.9/5/3.830 }{
			\edef\temp{
				\noexpand\addlegendimage{empty legend}
				\noexpand\addlegendentry{$n_{eff}=\Neff:$};
				\noexpand\addplot 
										{ + (1-\TA^2)^2 * \LA
											/ ( (1-\TA^2 * \LA)^2 	+ 4 * \TA^2 	* \LA *sin(deg(\Neff*pi*\Radius*2*pi*x*1e6/\CC))^2 )
										};
				\noexpand\addlegendentry{$D(\omega)$};
%				\noexpand\addplot 
%										{ + \TA^2 * ( (1- \LA)^2	+ 4 					* \LA *sin(deg(\Neff*pi*\Radius*2*pi*x*1e6/\CC))^2 )
%											/ ( (1-\TA^2 * \LA)^2 	+ 4 * \TA^2 	* \LA *sin(deg(\Neff*pi*\Radius*2*pi*x*1e6/\CC))^2 )
%										};
%				\noexpand\addlegendentry{$T(\omega)$};
				}
			\temp
		}
		
		\node [draw=black, fill=white, align=center] at (192.7,0.5) {$\tau=0.85$\\ $\gamma=0.9$\\ $R=\SI{5}{\um}$};
		
	\end{axis}
	
	\begin{axis}[
			width=\textwidth*0.75,%
			height=207pt,
			axis x line*= top,
			axis y line = none,
			xlabel = {Wavelength $\lambda$ [\si{\nm}]},
			samples=2,
			xmin= 190.95, xmax = 193,
			xtick = {193.41448903, 193.16524356, 192.91663964, 192.66867481, 192.421, 192.17465256, 191.92859027, 191.68315729, 191.43835121, 191.19416964, 190.95061019},
			scaled x ticks = manual:{}{ \pgfkeys{/pgf/fpu}\pgfmathparse{(1e-3*\CC/#1)} },
			/pgf/number format/1000 sep=,
		]
		\addplot[white, only marks] coordinates {(191,0.3) (192.95, 0.3)};
	\end{axis}
	
\end{tikzpicture}