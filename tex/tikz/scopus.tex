\tikzsetexternalprefix{tikz/}	% set subfolder
\tikzsetnextfilename{scopus}

\begin{tikzpicture}[baseline]
	
%	\begin{semilogyaxis}[
	\begin{axis}[
			title={Scopus documents analytics},
			xlabel={Year},
			ylabel={Documents},
			tick align=outside,
			width=\textwidth*0.75,%
			height=207pt,
			legend style={at={(0.5,-0.25)},anchor=north,legend columns=-1},
			/pgf/number format/1000 sep=,
		]
	  
		\addplot [semithick, blue, semithick, mark=*, mark size=1.5, mark options={solid}]
			table [x index=0,y index=1] {tikz/Scopus-NM.csv};
    \addlegendentry{ \scriptsize\textit{Neuromorphic}}	  
		\addplot [semithick, green, semithick, mark=*, mark size=1.5, mark options={solid}]
			table [x index=0,y index=1] {tikz/Scopus-NM-eng.csv};
    \addlegendentry{ \scriptsize\textit{Neuromorphic engeneering}}	  
		\addplot [semithick, orange, semithick, mark=*, mark size=1.5, mark options={solid}]
			table [x index=0,y index=1] {tikz/Scopus-NM-pho+opt.csv};
    \addlegendentry{ \scriptsize\textit{Neuromorphic AND (optics OR photonics)} }
    
	\end{axis}
%	\end{semilogyaxis}
\end{tikzpicture}

%ewcommand{\plotfile}[1]{
%    \pgfplotstableread[col sep=tab, header=true]{#1}{\table}
%    \pgfplotstablegetcolsof{#1}
%    \pgfmathtruncatemacro\numberofcols{\pgfplotsretval - 1}
%    \pgfplotsinvokeforeach{1,...,\numberofcols}{
%        \pgfplotstablegetcolumnnamebyindex{##1}\of{\table}\to{\colname}
%        \addplot [semithick, color##1, mark=*, mark size=1, mark options={solid}]%
%        		table [x index= 0, y index=##1] {#1};
%        \addlegendentryexpanded{ \colname }
%    }
%}